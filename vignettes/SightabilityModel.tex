%\documentclass[article]{jss}
\documentclass[nojss]{jss}
%\VignetteIndexEntry{Estimating Population Abundance Using Sightability Models: R SightabilityModel Package}

%%%%%%%%%%%%%%%%%%%%%%%%%%%%%%
%% declarations for jss.cls %%%%%%%%%%%%%%%%%%%%%%%%%%%%%%%%%%%%%%%%%%
%%%%%%%%%%%%%%%%%%%%%%%%%%%%%%

%% almost as usual
\author{John Fieberg\\Minnesota Department of Natural Resources}
\title{Estimating Population Abundance Using Sightability Models: \proglang{R} \pkg{SightabilityModel} Package}

%% for pretty printing and a nice hypersummary also set:
\Plainauthor{John Fieberg} %% comma-separated
\Plaintitle{Estimating Population Abundance Using Sightability Models: R SightabilityModel package} %% without formatting
\Shorttitle{SightabilityModel package} %% a short title (if necessary)

%% an abstract and keywords
\Abstract{
  This introduction to the \proglang{R}  \pkg{SightabilityModel} package is a slight modification of
  \citet{Fiebergjss}, published in the \textit{Journal of Statistical Software}.  Sightability
models are binary logistic-regression models used to estimate and
adjust for visibility bias in wildlife-population surveys
\citep{SS1989}. Estimation proceeds in 2 stages:  1) sightability
trials are conducted with marked individuals, and logistic
regression is used to estimate the probability of detection as a
function of available covariates (e.g., visual obstruction, group
size); 2) the fitted model is used to adjust counts (from future
surveys) for animals that were not observed. A modified
Horvitz-Thompson estimator is used to estimate abundance: counts of
observed animal groups are divided by their inclusion probabilites
(determined by plot-level sampling probabilities and the detection
probabilities estimated from stage 1). We provide a brief historical
account of the approach, clarifying  and documenting suggested
modifications to the variance estimators originally proposed by
\citet{SS1989}. We then introduce a new \proglang{R} package,
\pkg{SightabilityModel}, for estimating abundance using this
technique.  Lastly, we illustrate the software with a series of
examples using data collected from moose (\emph{Alces alces}) in
  northeastern Minnesota and mountain goats (\emph{Oreamnos americanus}) in Washington State. } \Keywords{abundance
estimation, Horvitz-Thompson, logistic regression, sightability
model, \proglang{R}, survey} \Plainkeywords{abundance estimation,
Horvitz-Thompson, logistic
regression, sightability model, R, survey} %% without formatting
%% at least one keyword must be supplied

%% publication information
%% NOTE: Typically, this can be left commented and will be filled out by the technical editor
%% \Volume{13}
%% \Issue{9}
%% \Month{September}
%% \Year{2004}
%% \Submitdate{2004-09-29}
%% \Acceptdate{2004-09-29}

%% The address of (at least) one author should be given
%% in the following format:
\Address{
  John Fieberg\\
  Biometrics Unit, Minnesota Department of Natural Resources\\
  5463-C W. Broadway, Forest Lake, MN\\
  E-mail: \email{john.fieberg@state.mn.us}\\
  URL: \url{http://fwcb.cfans.umn.edu/personnel/faculty/fieberg/index.htm/}
}
%% It is also possible to add a telephone and fax number
%% before the e-mail in the following format:
%% Telephone: +43/1/31336-5053
%% Fax: +43/1/31336-734

%% for those who use Sweave please include the following line (with % symbols):
%% need no \usepackage{Sweave.sty}

%% end of declarations %%%%%%%%%%%%%%%%%%%%%%%%%%%%%%%%%%%%%%%%%%%%%%%

\def\VAR{{\rm Var}\,}
\def\COV{{\rm Cov}\,}


\begin{document}
\section{Introduction}
Many wildlife populations are monitored using aerial surveys, in
which a random sample of spatial units (or plots) are flown and all
observed animals are counted.  Typically, not all animals in the
selected spatial units will be seen, and thus, population abundance estimators that
correct only for sampling will be biased low. \citet{SS1989}
developed a method for correcting aerial surveys for undetected
animals that is now widely
 used to estimate abundance of a variety of wildlife populations, including populations of
mountain goats \citep{Rice2009}, bighorn sheep \citep{Bodie1995},
Dall's sheep \citep{udevitz2006}, moose
\citep{Anderson1996,GiudFieb2011}, elk \citep{samuel1987, otten1993,
CoganDief1998, Lubow2002}, burrowing owls \citep{Manninginpress},
and various waterfowl species \citep{Pearse2008}.  The approach
proceeds in two stages. First, sightability trials are conducted
with marked individuals to identify variables (e.g., group size,
visual obstruction) that influence the probability that a group of
animals will be observed; these data are used to estimate parameters
of a logistic regression model. Second, the fitted logistic
regression model is used to adjust counts (from future surveys) for
animals that were not observed.

\citet{SS1989} suggested a modified Horvitz-Thompson estimator of
population size, in which animal counts are divided by their
estimated detection probabilities and plot-level sampling
probabilities.  They derived a complex expression for the variance
of the estimator that included 3 separate variance components, that
due to: 1) random sampling (of aerial plots);  2) random detection
(and missed detection) of independent groups of individuals; and  3)
estimation of parameters related to the detection process.  Although
\citet{SS1989} proposed estimators for these variance components,
these estimators were later shown to be biased. Consistent
estimators for the variance components were later developed as part
of a Ph.D. thesis \citep{Wong1996}, but these estimators have yet to
appear in the published literature or in available software.
Furthermore, the primary software used to implement the sightability
model estimator, \proglang{Program Aerial Survey}
\citep{Unsworth1999, Leban2007}, incorrectly implements
 \citet{SS1989}'s (biased) variance estimator  in the case of stratified survey designs
\citep{FiebGiud2008}.  Thus, there is a need to develop general
software to analyze data from wildlife sightability surveys.

The purpose of this paper is twofold:  1) we aim to provide a
historical account of the variance estimators associated with the
modified Horvitz-Thompson population abundance estimator originally
proposed by \citet{SS1989}; 2) we  introduce an \proglang{R} package
that implements \citet{SS1989}'s estimator along with
\citet{Wong1996}'s consistent variance component (and total
variance) estimators.  We demonstrate the use of the software with
data collected from moose (\emph{Alces alces}) in
  northeastern Minnesota and  mountain goats (\emph{Oreamnos americanus}) in Washington State.


\section{Modified Horvitz-Thompson estimator}

\subsection{Assumptions}
The basic assumptions of the sightability model approach, listed below, are taken verbatim from  \citet{SS1989}:
\begin{enumerate}
\item The population is geographically and demographically closed.
\item Groups of animals are independently observed.
\item Observed groups are completely enumerated and observed only
once.
\item The survey design for land units can be specified.
\item The probability of observing a group is known or can be
estimated.
\end{enumerate}

\subsection{Notation}

Below, we use the notation of \citet{thompson2002} and \citet{Wong1996}.  Let:


\begin{tabular}{r c p{14cm}}
$N$ & = & the number of spatial sampling units (hereafter ``plots'') in the population sampling frame.\\
$n$ & = & the number of plots in the sample. \\
$I_{i}$ & = & random variable, equal to 1 if the $i^{th}$ plot is selected and 0 otherwise.\\
$\pi_{i}$ & = & $\Prob(I_{i}=1)$, the probability that the $i^{th}$ plot is selected. \\
$\pi_{i,i'}$ & = & $\Prob(I_{i}=1, I_{i'}=1)$, the probability that both the $i^{th}$ and $i'^{th}$ plots are selected. \\
$M_{i}$ & = & the number of animal groups in the $i^{th}$ plot \{$i=1, 2,\ldots, N\}.$\\
$m_{i}$ & = & the number of animal groups observed or sighted in the $i^{th}$ sampled plot \{$i=1,\ldots,n\}.$\\
$y_{i,j}$ & = & the number of animals in the $j^{th}$ group located in the $i^{th}$ plot \{$j=1,2,\ldots,M_{i}$\}.\\
$Z_{i,j}$ & = & random variable, equal to 1 if the $j^{th}$ animal group in the $i^{th}$ plot is observed and 0 otherwise.  \\
$g_{i,j}$ & = & $\Prob(Z_{i,j}=1 | I_{i}=1)$, the probability that the $j^{th}$ animal group located in the $i^{th}$ sampled plot is observed (conditional on the $i^{th}$ plot
having been sampled).\\
$\theta_{i,j}$ & = & $1/g_{i,j}$, an `inflation factor' associated with the $j^{th}$ observed animal group in the $i^{th}$ sampled plot.\\
$\tau_{i}$ & = & the number of animals in the $i^{th}$ plot: $$\tau_{i} = \sum_{j=1}^{M_{i}}y_{i,j}$$ \\
$\tau$ & = & total population size: $$\tau = \sum_{i=1}^{N}\sum_{j=1}^{M_{i}}y_{i,j}=\sum_{i=1}^{N}\tau_{i}$$\\

\end{tabular}

\subsection{Sightability model approach}
The estimator developed in \citet{SS1989} involves the following
steps:
\begin{enumerate}
\item  A set of sightability trials are conducted by flying test plots, each containing a radiocollared animal.
  Although some animals are involved in
multiple test plot surveys, each survey attempt is treated as an
independent trial.  A suite of animal-specific covariates thought to
influence detection probabilities, $x_{i}$, are collected whenever
an animal is detected.  Similarly, if an animal is not detected when
flying a test plot, then telemetry is used to locate the individual
immediately after the test plot was surveyed.  If the individual is
still within the boundaries of the test plot, then the  same suite
of covariate information is collected and the observation is
included in the sightability data frame.  The end result is a set of
sightability (or detection) observations ($w_{i} = 1$ if the
radiocollared animal in the $i^{th}$
 sightability trial was seen and 0 otherwise) and a vector of covariate values $(x_{i})$ thought to influence detection
 probabilities.

\item  A logistic regression model (relating probability of detection to covariates) is built using data from step 1.  Let $\hat{\beta}$
= estimated regression parameters and  $\hat{\Sigma}$ = estimated variance-covariance matrix of $\hat{\beta}$.
\item  A new, operational survey, is conducted in which a random sample of plots are surveyed.  Within each
plot, individuals or groups of individuals are observed, and the
same covariates used to model detection probabilities in step 1 are
recorded.
\item The regression model from step 2 is used to estimate `inflation
factors', $\hat{\theta}_{i}$  = inverse probability of detection,
for each independent group in step 3;  these inflation factors
account for animals that were not detected during the operational
survey. \citet{SS1989} and \citet{Wong1996} suggest using:
\begin{equation}
\label{thetahat} \hat{\theta}_{i}=1+exp(-x_{i}^{\top}\hat{\beta}-
 x_{i}^{\top}\hat{\Sigma}x_{i}/2)
\end{equation}
Using large sample maximum likelihood
 theory and the result that $\hat{\beta}  \stackrel{d}{\to} N(\beta, \Sigma)$, \citet{SS1989} showed that this estimator will be consistent for $\frac{1}{g_{i,j}}$.


Similarly, \citet{SS1989} suggested the following expressions for
$\widehat{\VAR}(\hat{\theta}_{i})$ and
$\widehat{\COV}(\hat{\theta}_{i}, \hat{\theta}_{j})$, where $i$ and
$j$ index observations with two different sets of covariates $x_{i}$
and $x_{j}$:
\begin{eqnarray}
\label{varthetahat} \widehat{\VAR}(\hat{\theta}_{i}) & = & \exp(-2x_{i}^{\top} \hat{\beta}-2x_{i}^{\top}\hat{\Sigma}x_{i})(\exp(x_{i}^{\top}\hat{\Sigma}x_{i})-1)\\
\label{covthetahat} \widehat{\COV}(\hat{\theta}_{i},
\hat{\theta}_{j}) & = &
\exp(-(x_{i}+x_{j})^{\top}\hat{\beta}-(x_{i}+x_{j})^{\top}\hat{\Sigma}(x_{i}+x_{j})/2)(\exp(x_{i}^{\top}\hat{\Sigma}x_{j})-1)
\end{eqnarray}

These expressions, motivated by the asymptotic normality of
$\hat{\beta}$, will be needed to derive an expression for
$\widehat{\VAR}(\hat{\tau})$.

\item The total number of animals in the $i^{th}$ sampling unit is estimated by summing the number of independently observed animals (or groups of
animals) multiplied by their inflation factors (from step 4) across
all observed groups in the aerial unit: $$\hat{\tau}_{i} =
\sum_{j=1}^{M_{i}}Z_{i,j}y_{i,j}\hat{\theta}_{i,j}=\sum_{j=1}^{m_{i}}y_{i,j}\hat{\theta}_{i,j}
$$
\item The total number of animals in the study area, $\tau$ , is estimated by dividing values in step 5 by known probabilities of sampling each aerial unit
 to account for unsampled aerial units:

\begin{equation}
  \label{tauhat}
  \hat{\tau}= \sum_{i=i}^{N}\sum_{j=1}^{M_{i}} \frac{I_{i}Z_{i,j}y_{i,j}\hat{\theta}_{i,j}}{\pi_{i}} =\sum_{i=i}^{n}\sum_{j=1}^{m_{i}} \frac{y_{i,j}\hat{\theta}_{i,j}}{\pi_{i}}
\end{equation}
This estimator is a modified Horvitz-Thompson estimator because it
is constructed by dividing each observation by its estimated
inclusion
 probability - i.e., probability of the animal or group of animals being observed during the survey \citep{SS1989, thompson2002}.
\end{enumerate}

Note, the approach can also be applied to situations involving a
complete aerial census (by setting all $\pi_{i} = 1$).  In the same
way,
 one can apply the approach to non-randomly sampled plots (but inference should be restricted to these plots). Importantly,
 Horvitz-Thompson estimators are unbiased if inclusion probabilities are known.  When the inclusion probabilities are not known, the modified Horvitz-Thompson
 estimator will still be consistent, provided the estimators of the inclusion probabilities are consistent \citep{thompson2002}.



\section{Variance estimators}
The variance of $\hat{\tau}$ can be derived by applying the
 conditional variance formula  \citep{Casella1990}:
\begin{equation}
\label{condvar} \VAR(Y)=\E_{X}[\VAR(Y|X)]+\VAR_{X}[\E(Y|X)] .
\end{equation}
 To help understand the historical development of variance
estimators, it is helpful to distinguish between two cases:

\begin{description}
\item [Case A] Detection (i.e., sightability) probabilities are assumed known.
\item [Case B] Detection probabilities are estimated.
\end{description}

Following \citet{SS1989} and \citet{Wong1996}, we will refer to the
variance of $\hat{\tau}$ as $\VAR(\hat{\tau}_{\pi})$ for Case A and
$\VAR(\hat{\tau}_{LR})$ for Case B.

\subsection{Case A:  Detection parameters assumed known}
In the case where detection parameters are assumed known,
uncertainty arises from random sampling of aerial plots and also
from the random detection process, captured by the indicator
variables $I_{i}$ and $Z_{i,j}$, respectively. Conditioning on the
$I_{i}$, and applying Equation~\ref{condvar}, \citet{SS1989} derived
an expression for $\VAR(\hat{\tau}_{\pi})$ when detection
probabilities are known:
\begin{eqnarray}
\VAR(\hat{\tau}_{\pi})& = & \VAR_{I}(\E_{Z|I}(\hat{\tau}_{\pi})) + \E(\VAR_{Z|I}(\hat{\tau}_{\pi})) =  V_{s}+V_{d}, \mbox{ with} \nonumber \\
\label{SSvar} V_{s} & = &  \sum_{i=1}^{N}\frac{1-\pi_{i}}{\pi_{i}}\tau_{i}^{2}+\sum_{i\neq i'}^{N}\frac{\pi_{i,i'}-\pi_{i}\pi_{i'}}{\pi_{i}\pi_{i'}}\tau_{i}\tau_{i'}.\\
V_{d} & = &   \sum_{i=1}^{N}\frac{1}{\pi_{i}}\sum_{j=1}^{M_{i}}\frac{1-g_{i,j}}{g_{i,j}}y_{i,j}^2.
\end{eqnarray}

Thus, $\VAR(\hat{\tau}_{\pi})$ is  composed of two terms (or
components), the first of which is due to sampling $(V_{s})$;  the
second is due to the random detection process $(V_{d})$.
\citet{SS1989}  suggested estimators for these variance components,
which we will label $v_{s}$ and $v_{d}$:
\begin{eqnarray}
\label{vs} v_{s} & = & \sum_{i=1}^{n}\frac{1-\pi_{i}}{\pi_{i}^{2}}\hat{\tau}_{i}^{2}+\sum_{i\neq i'}^{n}\frac{\pi_{i,i'}-\pi_{i}\pi_{i'}}{\pi_{i,i'}\pi_{i}\pi_{i'}}\hat{\tau}_{i}\hat{\tau}_{i'} \\
\label{vd} v_{d} & = &
\sum_{i=1}^{n}\frac{1}{\pi_{i}^{2}}\sum_{j=1}^{m_{i}}\frac{1-g_{i,j}}{g_{i,j}^{2}}y_{i,j}^2,
\end{eqnarray}
where $\hat{\tau}_{i}=\sum_{j=1}^{m_{i}}y_{i,j}\hat{\theta}_{i,j}$.

\citet{thompsonSeb1996} also derived an expression for
$\VAR(\hat{\tau}_{\pi})$ using Equation~\ref{condvar}, but in
contrast to \citet{SS1989}, they derived their expression by
conditioning on the detection indicators, $Z_{i,j}$:
\begin{eqnarray}
 \VAR(\hat{\tau}_{\pi})& = & \E_{Z}(\VAR_{I|Z}(\hat{\tau}_{\pi})) + \VAR_{Z}(\E_{I|Z}(\hat{\tau}_{\pi})) =  V_{1}+V_{2}, \mbox{ with} \nonumber \\
\label{TSvar} V_{1} & = &  \E_{Z}(\sum_{i=1}^{N}\frac{1-\pi_{i}}{\pi_{i}}\hat{\tau}_{i}^{2}+\sum_{i \neq i'}^{N}\frac{\pi_{i,i'}-\pi_{i}\pi_{i'}}{\pi_{i}\pi_{i'}}\hat{\tau}_{i}\hat{\tau}_{i'})\\
V_{2} & = &
\sum_{i=1}^{N}\sum_{j=1}^{M_{i}}\frac{1-g_{i,j}}{g_{i,j}}y_{i,j}^2.
\end{eqnarray}

\citet{thompsonSeb1996} did not evaluate the expectation in the
expression for $V_{1}$ (Equation~\ref{TSvar}).  Nonetheless, they
recognized that the two variance decompositions give equivalent
expressions for $\VAR(\hat{\tau})$ (i.e., $V_{s}+V_{d} =
V_{1}+V_{2}$).  Using this equivalence, we can solve for $V_{1}$:
\begin{eqnarray}
V_{1} & =
&\sum_{i=1}^{N}\frac{1-\pi_{i}}{\pi_{i}}\tau_{i}^{2}+\sum_{i \neq
i'}^{N}\frac{\pi_{i,i'}-
\pi_{i}\pi_{i'}}{\pi_{i}\pi_{i'}}\tau_{i}\tau_{i'} +
\sum_{i=1}^{N}\frac{1-\pi_{i}}{\pi_{i}}\sum_{j=1}^{M_{i}}\frac{1-g_{i,j}}{g_{i,j}}y^{2}_{i,j}.
\end{eqnarray}


\citet{thompsonSeb1996} also showed that $v_{d}$ was an unbiased
estimator for $V_{d}$, but that
  $v_{s}$ was a biased estimator of
$V_{s}$. Lastly, \citet{thompsonSeb1996}  derived unbiased
estimators ($v_{1}$ and $v_{2}$) for the two terms in their variance
decomposition and, thus, provided an unbiased estimator for
$\VAR(\hat{\tau}_{\pi})$ when detection probabilities are assumed
known:
\begin{eqnarray}
v_{1} & = &  \sum_{i=1}^{n}\frac{1-\pi_{i}}{\pi_{i}^{2}}\hat{\tau}_{i}^{2}+\sum_{i \neq i'}^{n}\frac{\pi_{i,i'}-\pi_{i}\pi_{i'}}{\pi_{i,i'}\pi_{i}\pi_{i'}}\hat{\tau}_{i}\hat{\tau}_{i'}\\
v_{2} & = &
\sum_{i=1}^{n}\frac{1}{\pi_{i}}\sum_{j=1}^{m_{i}}\frac{1-g_{i,j}}{g_{i,j}^2}y_{i,j}^2.
\end{eqnarray}

[Note that $v_{s}$ is the same as $v_{1}$, but $v_{d}$ and $v_{2}$
differ by a factor of $1/\pi_{i}$.] Lastly, noting that
\citet{SS1989}'s estimator, $\widehat{\VAR}(\hat{\tau}_{\pi}) =
v_{s} + v_{d}$ is equal to $v_{1} + v_{d}$, \citet{thompsonSeb1996}
were able  to derive an unbiased estimate of $V_{s}$, which we will
label $v_{s}'$:
\begin{eqnarray}
v_{s}'& = &   v_{1}+v_{2}-v_{d} \nonumber \\
&= &
\sum_{i=1}^{n}\frac{1-\pi_{i}}{\pi_{i}^{2}}\hat{\tau}_{i}^{2}+\sum_{i
\neq
i'}^{n}\frac{\pi_{i,i'}-\pi_{i}\pi_{i'}}{\pi_{i,i'}\pi_{i}\pi_{i'}}\hat{\tau}_{i}\hat{\tau}_{i'}
-  \sum_{i=1}^{n}\frac{
1-\pi_{i}}{\pi_{i}^{2}}\sum_{j=1}^{m_{i}}\frac{1-g_{i,j}}{g_{i,j}^{2}}y_{i,j}^2
\end{eqnarray}

\citet{Wong1996} derived the same estimator (i.e., $v_{s}'$ for $V_{s}$) in her Ph.D. dissertation.

Interestingly, \citet{samuel1992} developed sightability estimators
for additional population quantities (e.g., sex ratios), and in
their Appendix, they suggested a new variance component estimator
for $V_{d}$ which turns out to be $v_{2}$.  Thus,
\citet{samuel1992}'s suggestion results in \emph{biased} estimators
for each individual variance component, but an \emph{unbiased}
estimate of $\VAR(\hat{\tau}_{\pi})$, equivalent to
\citet{thompsonSeb1996}'s estimator
$\widehat{\VAR}(\hat{\tau}_{\pi}) = v_{1}+v_{2}$, if detection
probabilities are assumed known.

\subsection{Case B:  Detection parameters are estimated}

\citet{SS1989} derived an expression for $\VAR(\hat{\tau}_{LR})$, by
first noting that:
\begin{equation}
\VAR(\hat{\tau}_{LR}) =
\VAR(\hat{\tau}_{\pi})+\E_{Z,I}\VAR(\hat{\tau}_{LR}|Z,I),
\end{equation}
where again, $\VAR(\hat{\tau}_{\pi})$ denotes the variance for the
case where detection parameters are assumed to be known. This last
term, $\E_{Z,I}\VAR(\hat{\tau}_{LR}|Z,I)$, captures uncertainty in
$\hat{\tau}_{LR}$ associated with the estimated (logistic
regression) detection parameters. Thus, \citet{SS1989} expressed the
total variance in terms of 3 components, sampling $[V_{s}]$,
sightability $[V_{d}]$, and that due to modeling the detection
probabilities $[V_{m}]$.  \citet{SS1989} further showed that:
\begin{eqnarray}
V_{m} & = & \E_{Z,I}\VAR(\hat{\tau}_{LR}|Z,I) \nonumber \\
&= &
\sum_{i=1}^{N}\frac{1}{\pi_{i}}\sum_{j=1}^{M_{i}}\frac{y_{i,j}^2}{\theta_{i,j}}\VAR(\hat{\theta}_{i,j})+
\sum_{i=1}^{N}\frac{1}{\pi_{i}}\sum_{j \neq j'}^{M_{i}}\frac{y_{i,j}y_{i,j'}}{\theta_{i,j}\theta_{i,j'}}\COV(\hat{\theta}_{i,j},\hat{\theta}_{i,j'}) \nonumber \\
  & & + \sum_{i \neq i'}^{N}\frac{\pi_{i,i'}}{\pi_{i}\pi_{i'}}\sum_{j=1}^{M_{i}}\sum_{j '=1}^{M_{i'}}\frac{y_{i,j}y_{i,j'}}{\theta_{i,j}\theta_{i,j'}}\COV(\hat{\theta}_{i,j},\hat{\theta}_{i',j'})
\end{eqnarray}

 \citet{SS1989} and \citet{Wong1996} derived the same estimator for $V_{m}$, which we will refer to as $v_{m}$:
\begin{eqnarray}
 v_{m} & = & \sum_{i=1}^{n}\frac{1}{\pi_{i}^{2}}\sum_{j=1}^{m_{i}}y_{i,j}^{2}
 \widehat{\VAR}(\hat{\theta}_{i,j}) + \sum_{i=1}^{n}\frac{1}{\pi_{i}^{2}}\sum_{j \neq j'}^{m_{i}}y_{i,j}y_{i,j'}\widehat{\COV}(\hat{\theta}_{i,j},\hat{\theta}_{i,j'}) \nonumber \\
\label{vm} & & \mbox{} + \sum_{i \neq
i'}^{n}\frac{1}{\pi_{i}\pi_{i'}}\sum_{j=1}^{m_{i}}\sum_{j'=1}^{m_{i}}y_{i,j}y_{i',j'}\widehat{\COV}(\hat{\theta}_{i,j},\hat{\theta}_{i',j'}),
\end{eqnarray}
where $\widehat{\VAR}(\hat{\theta}_{i,j})$ and
$\widehat{\COV}(\hat{\theta}_{i,j},\hat{\theta}_{i,j'})$ are given
by Equation~\ref{varthetahat} and Equation~\ref{covthetahat}.

\citet{SS1989} suggested estimating $\VAR(\hat{\tau}_{LR})$ by
combining $v_{m}$ with $v_{s}$ (Equation~\ref{vs}) and $v_{d}$
(Equation~\ref{vd}).  \citet{Wong1996} showed that while $v_{m}$ was
a consistent estimator of $V_{m}$, $v_{s}$ and $v_{d}$ were biased
(away from 0) when detection parameters were estimated. She derived
the following consistent estimators of $V_{s}$ and $V_{d}$,
$v_{s}^{*}$ and $v_{d}^{*}$ respectively,
 for the case where detection probabilities are estimated using logistic regression:
\begin{eqnarray}
v_{s}^{*} & = &
\sum_{i=1}^{n}\frac{1-\pi_{i}}{\pi_{i}^{2}}\hat{\tau}_{i}^{2}+\sum_{i
\neq
i'}^{n}\frac{\pi_{i,i'}-\pi_{i}\pi_{i'}}{\pi_{i,i'}\pi_{i}\pi_{i'}}\hat{\tau}_{i}\hat{\tau}_{i'}
-
 \sum_{i=1}^{n}\frac{1-\pi_{i}}{\pi_{i}^{2}}\sum_{j=1}^{m_{i}}y_{i,j}^2(\hat{\theta}_{i,j}^{2}-\hat{\theta}_{i,j}) \nonumber \\
& & \mbox{} -\sum_{i=1}^{n}\frac{1-\pi_{i}}{\pi_{i}^{2}}\sum_{j \neq j'}^{m_{i}}y_{i,j}y_{i,j'}\widehat{\COV}(\hat{\theta}_{i,j}, \hat{\theta}_{i,j'}) \nonumber \\
& & \mbox{} - \sum_{i \neq i'}^{n}\frac{\pi_{i,i'}-\pi_{i}\pi_{i'}}{\pi_{i,i'}\pi_{i}\pi_{i'}}\sum_{j=1}^{m_{i}}\sum_{j'=1}^{m_{i'}}y_{i,j}y_{i',j'}\widehat{\COV}(\hat{\theta}_{i,j},\hat{\theta}_{i',j'})\\
v_{d}^{*} & = &
\sum_{i=1}^{n}\frac{1}{\pi_{i}^{2}}\sum_{j=1}^{m_{i}}y_{i,j}^2(\hat{\theta}_{i,j}^{2}-\hat{\theta}_{i,j}-\widehat{\VAR}(\hat{\theta}_{i,j}))
\end{eqnarray}

Although these estimators have not been published in peer-reviewed
literature, they were used by \citet{Lubow2002} to analyze data
collected on elk (\emph{Cervus elaphus}) data in Rocky Mountain
National park (Colorado, USA).

\subsection{Confidence intervals}

The sampling distribution of $\hat{\tau}$ is often skewed right, and
as a result, confidence intervals constructed under an asymptotic
normality assumption often fail to have correct coverage rates
\citep{Wong1996, CoganDief1998}.  Using simulated data,
\citet{Wong1996} explored several alternative methods for
constructing confidence intervals.  Although the relative
performance of the various methods depended on the specifics of the
simulation, intervals constructed under the assumption that
($\hat{\tau}-T$) is lognormally distributed, where $T$ is the total
number of animals seen, tended to perform well across a range of
simulation scenarios. Using this approach, confidence limits are
formed using:
\begin{eqnarray}
 LCL & = & T + [(\hat{\tau}-T)/C]\sqrt{(1+cv^2)} \\
 UCL & = & T+[(\hat{\tau}-T)C]\sqrt{(1+cv^2)},
\end{eqnarray}
where $LCL$ and $UCL$ are the lower and upper confidence limits,
respectively, $cv^2 = var(\hat{\tau})/(\hat{\tau}-T)^2$, and $C =
\exp[z_{\alpha/2}\sqrt{\ln{(1+cv^2)}}]$.

\section{Package description and example applications}
The \pkg{SightabilityModel} package implements \citet{SS1989}'s
logistic regression sightability abundance estimator via the
\code{Sight.Est} function. This function requires, as arguments, the
original sightability trial data frame and an operational survey
data frame. Alternatively, one can specify a pre-fitted model (using
$\hat{\beta}$ and $\hat{\Sigma}$ as arguments to \code{Sight.Est}).
Covariates used in the sightability model must be present in both
data frames (with identical naming conventions in both).  The
operational data frame also requires variables named \code{total}
(containing the animal counts, $y_{i,j}$) and \code{subunit} (a
sample plot identifier).  Most operational surveys employ stratified
random sampling designs to select survey plots (as a means of
increasing precision), with plots allocated to strata based on their
expected animal densities \citep[e.g., `low', `medium', and
`high';][]{FiebLenarz}. Thus, the operational data frame must also
include a variable named \code{stratum}, which serves as a stratum
identifier (\code{stratum} should take on a single value for
non-stratified surveys). Lastly, the user must supply a data frame
containing sampling information, including \code{nh} (number of
sampled plots in each stratum), \code{Nh} (number of population
units in each stratum), and a variable named \code{stratum} (taking
on the same values as the \code{stratum} variable in the operational
data frame); for non-stratified survey designs this data frame will
contain a single record.

      The \code{Sight.Est} function
fits a logistic regression model to the sightability data frame and
applies the model to data from the operational survey to estimate
population abundance, $\hat{\tau}$.  The variance of  $\hat{\tau}$
can be estimated using $\widehat{\VAR}(\hat{\tau}_{LR}) =
v_{1}+v_{2} + v_{m}$ as suggested by \citet{samuel1992} or using
\citet{Wong1996}'s estimator $\widehat{\VAR}(\hat{\tau}_{LR}) =
v_{s}^{*} +v_{d}^{*}+v_{m}$ (recommended).  Alternatively, one can
use a nonparametric bootstrap to aid in the estimation of the
variance components (see Example 5 in the next section).


\subsection{ Example 1:  Application to moose survey data}
We illustrate the software using data collected from moose
(\emph{Alces alces}) in northeastern Minnesota \citep{GiudFieb2011}.
Sightability trial data were collected from 2005-2007 (124 trials
were conducted and the results are captured in the \code{exp.m} data
frame); operational survey data from 2004-2007 are also included in
the package (\code{obs.m} data frame).  Lastly, sampling information
from the 2004-2007 operational surveys is captured in the
\code{sampinfo.m} data frame.

\begin{Schunk}
\begin{Sinput}
>  options(prompt = "R> ", continue = "+  ", width = 70, useFancyQuotes = FALSE)
\end{Sinput}
\end{Schunk}
\begin{Schunk}
\begin{Sinput}
R>  library("SightabilityModel")
R>  data("obs.m")
R>  data("exp.m")
R>  data("sampinfo.m")
\end{Sinput}
\end{Schunk}

\begin{Schunk}
\begin{Sinput}
R> exp.m[1:5, ] # first 5 observations
\end{Sinput}
\begin{Soutput}
   year observed voc grpsize
37 2005        1  20       4
38 2005        1  85       2
39 2005        0  80       1
40 2005        0  75       1
41 2005        0  70       1
\end{Soutput}
\end{Schunk}
The experimental data frame contains 4 variables: \code{year} (year
of the test trial), \code{observed} (equal to 1 if the moose was
observed and 0 otherwise), \code{voc} (amount of screening cover
within 4 animal lengths of the first animal seen), and
\code{grpsize} (number of animals associated with the radiocollared
animal, i.e., cluster size).  Each row represents an independent
sightability trial, with \code{observed} representing the random
variables ($w_{i})$.
\begin{Schunk}
\begin{Sinput}
R> obs.m[1:5, ]
\end{Sinput}
\begin{Soutput}
  year stratum subunit total cows calves bulls unclass voc grpsize
1 2004       1     140     2    1      1     0       0  90       2
2 2004       1     140     1    0      0     1       0  90       1
3 2004       1     140     2    0      0     2       0  75       2
4 2004       1     140     3    1      0     2       0   0       3
5 2004       1     180     2    2      0     0       0  85       2
\end{Soutput}
\end{Schunk}
Each record in the operational data frame corresponds to an
independently sighted group of moose, with variables that capture
animal-specific covariates, $x_{i} = $ (\code{voc, grpsize}), for
potential inclusion in the detection model.  The data frame also
contains plot-level information (\code{stratum} = stratum identifer,
\code{subunit} = sample plot identifier). Lastly, \code{total} gives
the total number of animals observed in each independently sighted
group (note:  this variable is redundant with \code{grpsize};  the
latter variable is included to allow one to model detection
probabilities as a function of group size).

Lastly, the sampinfo.m data frame contains sampling information
associated with the operational surveys conducted in years
2004-2007. Specifically, it contains the number of sampled $(n_{h})$
and population $(N_{h})$ units in each stratum:
\begin{Schunk}
\begin{Sinput}
R> sampinfo.m
\end{Sinput}
\begin{Soutput}
   year stratum  Nh nh
1  2004       1 178 10
2  2004       2 127 14
3  2004       3  31  6
4  2005       1 238 12
5  2005       2 180 18
6  2005       3  35  6
7  2006       1 238 18
8  2006       2 180 14
9  2006       3  35  5
10 2007       1 248 16
11 2007       2 170 18
12 2007       3  35  6
\end{Soutput}
\end{Schunk}
The \code{Sight.Est} function is used both to fit a specified
logistic regression model and estimate population abundance in a
single step. Below, we illustrate the code using operational survey
data collected in 2004 only, modeling detection probabilities as a
function of visual obstruction (i.e., \code{voc}; note, an
intercept-only model, in which  detection probabilities are assumed
to be constant, can be specified using \code{observed~1} in the
\code{Sight.Est} function):
\begin{Schunk}
\begin{Sinput}
R> est.2004 <- Sight.Est(observed ~ voc, odat = subset(obs.m,
+      year == 2004), sdat = exp.m, sampinfo = subset(sampinfo.m,
+      year == 2004))
\end{Sinput}
\end{Schunk}
By default, \citet{Wong1996}'s estimators are used to estimate
$\VAR(\hat{\tau}_{LR})$, and confidence intervals are formed under
the assumption that ($\hat{\tau}-T$) is lognormally distributed.
The print function provides information on the fitted sightability
model and sampling statistics. In addition, it provides the point
estimate and confidence interval as well as estimates of each of the
3 variance components.
\begin{Schunk}
\begin{Sinput}
R> print(est.2004)
\end{Sinput}
\begin{Soutput}
Call:
Sight.Est(form = observed ~ voc, sdat = exp.m, odat = subset(obs.m,
    year == 2004), sampinfo = subset(sampinfo.m, year == 2004))

------------------- SIGHTABILITY MODEL ---------------------

Call:  glm(formula = form, family = binomial(), data = sdat)

Coefficients:
(Intercept)          voc
    1.75993     -0.03479

Degrees of Freedom: 123 Total (i.e. Null);  122 Residual
Null Deviance:      171.6
Residual Deviance: 147.4    AIC: 151.4

----------------- Population Survey data  ----------------

 Stratum Sampling Information
  stratum nh  Nh
1       1 10 178
2       2 14 127
3       3  6  31

 Number of animals seen in each stratum
  1   2   3
 68 248 202

-------------- POPULATION ESTIMATE ( 95 % CI) ----------------

[1] tau.hat =  13,096 ;   95 % CI = (  8,549 ,  21,430 )


------------------  SE(tau.hat) --------------------------------
Variance method: [1] "WONG"
   SE
3,117

-------------- Variance Components -------------------

  VarSamp  VarSight    VarMod
3,033,952   873,246 5,810,441
\end{Soutput}
\end{Schunk}
Note: the variance component estimates provide useful information
for improving future surveys.  In particular, the first and third
variance components (sampling uncertainty and parameter uncertainty
associated with the detection model, respectively) are under control
of the observer. Sampling uncertainty can be reduced by surveying a
larger number of aerial plots or by implementing a more efficient
sampling design \citep{FiebLenarz}, whereas parameter uncertainty
can be reduced by conducting more sightability trials. The large
variance component associated with model parameters (i.e.,
\code{VarMod}), suggests that conducting additional sightability
trials would be useful, particularly since the benefits would be
realized in multiple years (i.e., all estimates using the same
sightability model will be improved by reducing this variance
component).

For a more concise summary, one can use the \code{summary} function,
which returns the point estimate and confidence interval:
\begin{Schunk}
\begin{Sinput}
R> summary(est.2004)
\end{Sinput}
\begin{Soutput}
[1] tau.hat =  13,096 ;   95 % CI = (  8,549 ,  21,430 )
\end{Soutput}
\end{Schunk}

\subsection{Example 2a:  Correlated stratum-specific estimates}
Stratified random sampling is often used to select plots in aerial
surveys. Unlike traditional survey estimates that employ stratified
random sampling, stratum-specific abundance estimates will be
correlated when a common detection model is used to correct counts
in all strata \citep{FiebGiud2008}. Unfortunately, many authors have
mistakenly assumed stratum-specific variances will sum to give the
total variance; ignoring the correlation among stratum-specific
estimates will typically result in an underestimate of the total
variance \citep{FiebGiud2008}.  This next example highlights this
point.

We begin by analyzing 2004 survey data for each stratum separately,
storing the results in a matrix named \code{tau.hats}:
\begin{Schunk}
\begin{Sinput}
R> tau.hats <- matrix(NA, 3, 5)
R> rownames(tau.hats) <- c("Stratum 1", "Stratum 2", "Stratum 3")
R> for(i in 1:3){
+    tempsamp <- sampinfo.m[i, ]
+    tempobs <- obs.m[obs.m$year == 2004 & obs.m$stratum == i, ]
+    temp <- Sight.Est(observed ~ voc, odat = tempobs, sdat = exp.m,
+        sampinfo = tempsamp)
+    tau.hats[i, ] <- temp$est
+  }
R> colnames(tau.hats) <- names(temp$est)
R> tau.hats<-round(tau.hats, 0)
R> print(format(tau.hats, big.mark = ","),  quote = FALSE)
\end{Sinput}
\begin{Soutput}
          tau.hat   VarTot    VarSamp   VarSight  VarMod
Stratum 1     4,152 2,892,280 1,555,085   487,767   849,429
Stratum 2     5,098 1,389,226   797,124   191,047   401,055
Stratum 3     3,846 1,679,580   681,744   194,431   803,405
\end{Soutput}
\end{Schunk}
We then compare the sum of the stratum-specific abundance estimates
and variance component estimates to those obtained by applying the
\code{Sight.Est} function once (i.e., to data from all strata).
\begin{Schunk}
\begin{Sinput}
R> est.2004 <- Sight.Est(observed ~ voc, odat = subset(obs.m,
+      year == 2004), sdat = exp.m, sampinfo = subset(sampinfo.m,
+      year == 2004))
R> print(format(round(est.2004$est, 0), big.mark = ","), quote = FALSE)
\end{Sinput}
\begin{Soutput}
  tau.hat    VarTot   VarSamp  VarSight    VarMod
   13,096 9,717,638 3,033,952   873,246 5,810,441
\end{Soutput}
\begin{Sinput}
R> naive.tau.hats<-round(apply(tau.hats[1:3, ], 2, sum), 0)
R> print(format(naive.tau.hats, big.mark = ","), quote = FALSE)
\end{Sinput}
\begin{Soutput}
  tau.hat    VarTot   VarSamp  VarSight    VarMod
   13,096 5,961,086 3,033,953   873,245 2,053,889
\end{Soutput}
\end{Schunk}

Note that the point estimate formed by summing the stratum-specific
abundance estimates is correct, but the sum of the stratum-specific
variances is too small (5,961,086 compared to 9,717,638).  The
difference is due to the positive covariance among the
stratum-specific estimates as a result of applying the same
detection model to each stratum (note the variance components due to
sampling and sightabilty are independent across strata).

Thus, if stratum-specific estimates are of interest, they can be
obtained using separate calls to \code{Sight.est}, but the variance
estimate for the total population size (i.e., aggregated across
strata) requires that all data be processed simultaneously (e.g.,
using an additional call to \code{Sight.Est} with all of the
operational survey data included).

\subsection{Example 2b:  Non-independent population estimates}
Often, management agencies are interested in changes in population
size (e.g., from one year to the next).  Similar to stratum-specific
estimates, population estimates will not be independent across years
if they are formed using the same sightability model. Typically,
estimates will exhibit a positive covariance, making tests (for a
difference between years) conservative if estimates are assumed to
be independent.

The \code{vardiff} function can be used to estimate the variance of
a difference between two population estimates, while accounting for
any covariance between the estimates due to using the same
sightability model. The function takes as arguments 2 sightability
model objects created from separate calls to the \code{Sight.Est}
function.
\begin{Schunk}
\begin{Sinput}
R> est.2006 <- Sight.Est(observed ~ voc, odat = subset(obs.m,
+      year == 2006), sdat = exp.m, subset(sampinfo.m,
+      year == 2006))
R> est.2007 <- Sight.Est(observed ~ voc, odat = subset(obs.m,
+      year == 2007), sdat = exp.m, subset(sampinfo.m,
+      year == 2007))
R> vdiff<-vardiff(est.2006, est.2007)
R> print(format(vdiff, nsmall = 0, big.mark = ","), quote = FALSE)
\end{Sinput}
\begin{Soutput}
     Var(difference)
[1,] 2,562,188
\end{Soutput}
\end{Schunk}

\begin{Schunk}
\begin{Sinput}
R> naive<-est.2006$est[2] + est.2007$est[2]
R> print(format(naive, nsmall = 0, big.mark = ","), quote = FALSE)
\end{Sinput}
\begin{Soutput}
   VarTot
3,649,057
\end{Soutput}
\end{Schunk}
As expected, the estimates of moose abundance in 2006 and 2007
exhibit a positive covariance due to using the same sightability
model to correct for detection. Thus, the naive estimate (formed by
summing the individual variances as though the two estimates were
independent) is too large.  We therefore recommend using the
\code{vardiff} function whenever estimating differences in
population size across years. We also provide a function
(\code{varlog.lam}) to calculate the variance of the log rate of
change between two population estimates (i.e.,
$var(log(\hat{\tau}_{2}/\hat{\tau}_{1}))$.

Lastly, it is interesting to note that naive estimates of variance
(assuming independence) were too small in Example 2a, but too large
in Example 2b. In both cases, individual estimates were positively
correlated because they were formed using the same sightability
model.  These results follow from the fact that Example 1a involves
estimation of a sum [with $var(x+y) = var(x) + var(y) +2cov(x,y)$]
whereas Example 1b involves estimation of a difference [with
$var(x-y) = var(x)+var(y)-2cov(x,y)$].

\subsection{Example 3:  Non-linear detection functions}
\citet{GiudFieb2011} considered models in which the logit detection
probability varied non-linearly as a function of \code{voc}. In this
next example, we show how the \code{ns} function in the splines
package of Program \proglang{R} \citep{RCDT2010} can be used to
allow for non-linear sightability models (on the logit scale). To do
so, we need to create the spline basis functions (in both
experimental and operational data frames) first before calling
\code{Sight.Est}. Below, we allocate 3 degrees of freedom to model
the effect of \code{voc} on the probability of detection.
\begin{Schunk}
\begin{Sinput}
R> library("splines")
\end{Sinput}
\end{Schunk}

\begin{Schunk}
\begin{Sinput}
R> exp.m$voc.ns <- ns(exp.m$voc, df = 3)
R> obs.m$voc.ns <- predict(exp.m$voc.ns, obs.m$voc)
R> ns.est <- Sight.Est(observed ~ voc.ns, odat = subset(obs.m,
+      year == 2004), sdat = exp.m, subset(sampinfo.m,
+      year == 2004))
R> ns.est$sight
\end{Sinput}
\begin{Soutput}
Call:  glm(formula = form, family = binomial(), data = sdat)

Coefficients:
(Intercept)      voc.ns1      voc.ns2      voc.ns3
      2.858       -2.056       -6.902       -2.565

Degrees of Freedom: 123 Total (i.e. Null);  120 Residual
Null Deviance:      171.6
Residual Deviance: 145.7    AIC: 153.7
\end{Soutput}
\begin{Sinput}
R> print(format(round(ns.est$est, 0),  big.mark = ","), quote = FALSE)
\end{Sinput}
\begin{Soutput}
   tau.hat     VarTot    VarSamp   VarSight     VarMod
    12,585 10,744,879  2,559,523    608,521  7,576,835
\end{Soutput}
\end{Schunk}

\subsection{Example 4:  Using multi-model inference techniques}

Many sightability studies collect a wide array of covariate data for
relatively few numbers of sightability trials.  Developing
predictive models that perform well on future data can be
challenging in these situations \citep{GiudFieb2011}. Model
averaging can serve as a form of shrinkage, and thus may help to
improve predictive accuracy when applying the model to new data
\citep{burnham2002model, GiudFieb2011}. \citet{burnham2002model}
describe a popular approach for performing model averaging in
wildlife research, including the calculation of unconditional
variance-covariance matrices that attempt to account for model
selection uncertainty.

Model-averaged parameter estimates and unconditional
variance-covariance matrices can be used to estimate abundance, by
passing these values as arguments to the \code{Sight.Est} function.
For illustrative purposes, we consider multi-model inference results
from \citet{Rice2009}, applied to an operational survey of mountain
goats in Olympic National Park, Washington. Model-averaged
regression parameters (for covariates \code{GroupSize},
\code{Terrain}, and \code{pct.VegCover}), and the corresponding
unconditional variance-covariance matrix from \citet{Rice2009} are
stored in a list named g.fit:
\begin{Schunk}
\begin{Sinput}
R> data("g.fit")
\end{Sinput}
\end{Schunk}
\begin{Schunk}
\begin{Sinput}
R> g.fit
\end{Sinput}
\begin{Soutput}
$beta.g
[1]  1.385  0.158 -1.138 -0.012

$varbeta.g
          [,1]      [,2]      [,3]     [,4]
[1,]  0.160801 -0.013000 -0.090000 -1.0e-03
[2,] -0.013000  0.003721 -0.001000  0.0e+00
[3,] -0.090000 -0.001000  0.161604  0.0e+00
[4,] -0.001000  0.000000  0.000000  8.1e-05
\end{Soutput}
\end{Schunk}
\begin{Schunk}
\begin{Sinput}
R> data("gdat")
\end{Sinput}
\end{Schunk}
\begin{Schunk}
\begin{Sinput}
R> gdat[1:5, ]
\end{Sinput}
\begin{Soutput}
  GroupSize Terrain pct.VegCover stratum total subunit
1         1       0        38.00       1     1      69
2         1       0        38.00       1     1      69
3         1       0         3.25       1     1      69
4         1       0         3.25       1     1      69
5         1       0         1.77       1     1      69
\end{Soutput}
\end{Schunk}
We create a data frame to hold the sampling information for the
operational survey:
\begin{Schunk}
\begin{Sinput}
R> sampinfo<-data.frame(nh = c(6, 23, 11), Nh =
+  c(6, 27, 65), stratum=c(1,2,3))
\end{Sinput}
\end{Schunk}
Although 3 strata were surveyed, no mountain goats were observed in
the 11 low stratum plots.
\begin{Schunk}
\begin{Sinput}
R> table(gdat$stratum)
\end{Sinput}
\begin{Soutput}
 1  2
50 27
\end{Soutput}
\end{Schunk}
Thus, we only supply the first two records of \code{sampinfo} when
we call \code{Sight.Est}. We also need to specify \code{bet=beta.g}
and \code{varbet=varbeta.g} to indicate that we are supplying our
own regression model (fit outside of \code{Sight.Est}):
\begin{Schunk}
\begin{Sinput}
R> goat.est<-Sight.Est(observed ~ GroupSize + Terrain + pct.VegCover,
+      odat = gdat,  sampinfo = sampinfo[1:2, ], bet = g.fit$beta.g,
+      varbet = g.fit$varbeta.g)
R> print(goat.est)
\end{Sinput}
\begin{Soutput}
Call:
Sight.Est(form = observed ~ GroupSize + Terrain + pct.VegCover,
    odat = gdat, sampinfo = sampinfo[1:2, ], bet = g.fit$beta.g,
    varbet = g.fit$varbeta.g)

------------------- SIGHTABILITY MODEL ---------------------
$bet
[1]  1.385  0.158 -1.138 -0.012

$varbet
          [,1]      [,2]      [,3]     [,4]
[1,]  0.160801 -0.013000 -0.090000 -1.0e-03
[2,] -0.013000  0.003721 -0.001000  0.0e+00
[3,] -0.090000 -0.001000  0.161604  0.0e+00
[4,] -0.001000  0.000000  0.000000  8.1e-05

$note
[1] "User supplied regression model"


----------------- Population Survey data  ----------------

 Stratum Sampling Information
  stratum nh Nh
1       1  6  6
2       2 23 27

 Number of animals seen in each stratum
  1   2
124  56

-------------- POPULATION ESTIMATE ( 95 % CI) ----------------

[1] tau.hat =  230 ;   95 % CI = ( 205 ,  293 )


------------------  SE(tau.hat) --------------------------------
Variance method: [1] "WONG"
SE
20

-------------- Variance Components -------------------

 VarSamp VarSight   VarMod
     137      168       82
\end{Soutput}
\end{Schunk}

\subsection{Example 5:  Use of the bootstrap to estimate variance components}
\citet{FiebGiud2008}  used simulations to explore the reliability of
the estimators of $\VAR(\theta_{i,j})$ and $\COV(\theta_{i,j},
\theta_{i',j'})$ given in Equation~\ref{varthetahat} and
\ref{covthetahat}. They found that these variance and covariance
terms were generally
 underestimated when a small number of sightability trials were used to estimate $\beta$ (e.g., $< 100$), and
suggested a non-parametric bootstrap  (e.g., resampling sightability
data with replacement) might be useful for estimating these terms
\citep[see also e.g.,\,][]{Wong1996, CoganDief1998}.

If the user specifies \code{Vm.boot = TRUE} when calling the
\code{Sight.Est} function, $\widehat{\VAR}(\theta_{i,j})$ and
$\widehat{\COV}(\theta_{i,j}, \theta_{i',j'})$ will be estimated by
applying a non-parametric bootstrap to the sightability data frame.
Specifically, the logistic regression model is fit to \code{nboot}
bootstrap data sets and $\VAR(\theta_{i,j})$ and $\COV(\theta_{i,j},
\theta_{i',j'})$ are estimated using  empirical variance/covariances
across bootstrap replicates.  These estimates are then plugged into
formulas used for estimating $V_{s}, V_{d}$, and $V_{m}$.

In this example, we compare analytical and bootstrap estimates of
variance (with 10,000 bootstrap replicates), considering only the
subset of moose experimental data collected  in 2005 (representing a
total of 39 sightability trials).
\begin{Schunk}
\begin{Sinput}
R> analytical.est <- Sight.Est(observed ~ voc, odat = subset(obs.m,
+     year == 2004), sdat = subset(exp.m, year == 2005), subset(sampinfo.m,
+     year == 2004), method = "Wong", logCI = T, alpha = 0.05,
+     Vm.boot = FALSE)
R> print(format(round(analytical.est$est, 0), big.mark = ","), quote = FALSE)
\end{Sinput}
\begin{Soutput}
  tau.hat    VarTot   VarSamp  VarSight    VarMod
    8,289 3,266,683 1,184,332   183,050 1,899,301
\end{Soutput}
\begin{Sinput}
R> boot.est <- Sight.Est(observed ~ voc, odat = subset(obs.m,
+     year == 2004), sdat=subset(exp.m, year == 2005), subset(sampinfo.m,
+     year == 2004), method = "Wong", logCI = T, alpha = 0.05,
+     Vm.boot = TRUE, nboot = 10000)
R> print(format(round(boot.est$est, 0), big.mark = ","), quote = FALSE)
\end{Sinput}
\begin{Soutput}
  tau.hat    VarTot   VarSamp  VarSight    VarMod
    8,289 5,553,521 1,118,655   146,832 4,288,035
\end{Soutput}
\end{Schunk}

As expected, the variance estimate calculated using the bootstrap is
larger than the analytical estimate (constructed using  the
estimators in Equation~\ref{varthetahat} and
Equation~\ref{covthetahat}).

\section{Future work}
Another potential concern with applying the sightability model
approach is that group size,  a key covariate in many detection
models, is often estimated with error
 \citep{CoganDief1998,Walsh2009}. In addition to problems associated with estimating regression parameters when covariates are subject to measurement error,
  animal counts, $y_{i,j}$, will usually be too small.
  \citet{Walsh2009} suggested a double observer approach to correct for this problem.  We hope to incorporate this option in a future version of the \pkg{SightabilityModel}
  package.  In addition, we are currently exploring Bayesian implementations
  that utilize data augmentation, similar to methods developed for
  analyzing mark-recapture data with individual covariates
  \citep{Royle,RoyleDorazio}.  Lastly, we hope to incorporate
 sightability estimators of other population quantities (e.g.,
 calf:cow ratios) as described by \citet{samuel1992}.


%%\section{Future work}
%%Group size stuff could be included, overall bootstrap, sex ratios, etc.
\section*{Acknowledgements}
This work was supported by the Minnesota Department of Natural
Resources.  We thank John Giudice, Mark Lenarz, Kurt Jenkins, and
Paul Griffin for helpful comments and for testing computer code.  We
thank Duane Diefenbach, two anonymous reviewers, and the associate
editor for suggestions that helped improve the clarity of the
manuscript and computer code.

\bibliographystyle{jss}
\bibliography{vignette}

\end{document}
